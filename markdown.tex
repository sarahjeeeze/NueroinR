\documentclass[]{article}
\usepackage{lmodern}
\usepackage{amssymb,amsmath}
\usepackage{ifxetex,ifluatex}
\usepackage{fixltx2e} % provides \textsubscript
\ifnum 0\ifxetex 1\fi\ifluatex 1\fi=0 % if pdftex
  \usepackage[T1]{fontenc}
  \usepackage[utf8]{inputenc}
\else % if luatex or xelatex
  \ifxetex
    \usepackage{mathspec}
  \else
    \usepackage{fontspec}
  \fi
  \defaultfontfeatures{Ligatures=TeX,Scale=MatchLowercase}
\fi
% use upquote if available, for straight quotes in verbatim environments
\IfFileExists{upquote.sty}{\usepackage{upquote}}{}
% use microtype if available
\IfFileExists{microtype.sty}{%
\usepackage{microtype}
\UseMicrotypeSet[protrusion]{basicmath} % disable protrusion for tt fonts
}{}
\usepackage[margin=1in]{geometry}
\usepackage{hyperref}
\hypersetup{unicode=true,
            pdftitle={Neurohacking},
            pdfborder={0 0 0},
            breaklinks=true}
\urlstyle{same}  % don't use monospace font for urls
\usepackage{graphicx,grffile}
\makeatletter
\def\maxwidth{\ifdim\Gin@nat@width>\linewidth\linewidth\else\Gin@nat@width\fi}
\def\maxheight{\ifdim\Gin@nat@height>\textheight\textheight\else\Gin@nat@height\fi}
\makeatother
% Scale images if necessary, so that they will not overflow the page
% margins by default, and it is still possible to overwrite the defaults
% using explicit options in \includegraphics[width, height, ...]{}
\setkeys{Gin}{width=\maxwidth,height=\maxheight,keepaspectratio}
\IfFileExists{parskip.sty}{%
\usepackage{parskip}
}{% else
\setlength{\parindent}{0pt}
\setlength{\parskip}{6pt plus 2pt minus 1pt}
}
\setlength{\emergencystretch}{3em}  % prevent overfull lines
\providecommand{\tightlist}{%
  \setlength{\itemsep}{0pt}\setlength{\parskip}{0pt}}
\setcounter{secnumdepth}{0}
% Redefines (sub)paragraphs to behave more like sections
\ifx\paragraph\undefined\else
\let\oldparagraph\paragraph
\renewcommand{\paragraph}[1]{\oldparagraph{#1}\mbox{}}
\fi
\ifx\subparagraph\undefined\else
\let\oldsubparagraph\subparagraph
\renewcommand{\subparagraph}[1]{\oldsubparagraph{#1}\mbox{}}
\fi

%%% Use protect on footnotes to avoid problems with footnotes in titles
\let\rmarkdownfootnote\footnote%
\def\footnote{\protect\rmarkdownfootnote}

%%% Change title format to be more compact
\usepackage{titling}

% Create subtitle command for use in maketitle
\providecommand{\subtitle}[1]{
  \posttitle{
    \begin{center}\large#1\end{center}
    }
}

\setlength{\droptitle}{-2em}

  \title{Neurohacking}
    \pretitle{\vspace{\droptitle}\centering\huge}
  \posttitle{\par}
    \author{}
    \preauthor{}\postauthor{}
    \date{}
    \predate{}\postdate{}
  

\begin{document}
\maketitle

\begin{verbatim}
library("oro.dicom")

setwd("C:/Neurohacking/Neurohacking_data/BRAINIX/DICOM/FLAIR")
slice=readDICOM("IM-0001-0011.dcm")
class(slice)
#explore the data
names(slice)
class(slice$hdr)
class(slice$hdr[[1]])
class(slice$img)
class(slice$img[[1]])
dim(slice$img[[1]])

#view the image
d=dim(t(slice$img[[1]]))
image(1:d[1],1:d[2],t(slice$img[[1]]),col=gray(0:64/64))

#look at the actual numbers

slice$img[[1]][101:105,121:125]

#plot entire thing - make density instead of count
hist(slice$img[[1]][,],breaks=50 ,xlab="FLAIR",prob=T, col=rgb(0,1,1,1/4),main="")

#explore header info

hdr=slice$hdr[[1]]
names(hdr)
hdr$name
#lookk at resoltuion
hdr[hdr$name == "PixelSpacing", "value"]
#flip angle
hdr[hdr$name == "FlipAngle",]

#change directory

setwd("C:/Neurohacking/Neurohacking_data/BRAINIX/DICOM")
all_slices_T1=readDICOM("T1/")
dim(all_slices_T1$img[[11]])
hdr=all_slices_T1$hdr[[11]]
\end{verbatim}

\begin{figure}
\centering
\includegraphics{C:/Neurohacking/Neurohacking_data/BRAINIX/NIfTI/Rplotbrain.jpeg}
\caption{dicom slice}
\end{figure}

\begin{verbatim}
#nifti stuff
setwd("C:/Neurohacking/Neurohacking_data/BRAINIX/DICOM")
all_slices_T1=readDICOM("T1/")
dim(all_slices_T1$img[[11]])
hdr=all_slices_T1$hdr[[11]]

#convert from dicom to nifti using dicom2nifti

nii_T1=dicom2nifti(all_slices_T1)
#check how man slices and pixels
d=dim(nii_T1); d; class(nii_T1)
#pick out 11th slice
image(1:d[1],1:d[2],nii_T1[,,11],col=gray(0:64/64),xlab="",ylab="")

setwd("C:/Neurohacking/Neurohacking_data/BRAINIX/NIfTI")
fname="Output_3D_File"
#look for the files
writeNIfTI(nim=nii_T1,filename=fname)
list.files(getwd(),patter="Output_3D_File")
list.files(getwd(),pattern="T")
#don't reorient 
nii_T2=readNIfTI("T2.nii.gz",reorient=FALSE)
dim(nii_T2)
print({nii_T1 = readNIfTI(fname=fname)})   
image(1:d[1],1:d[2],nii_T1[,,11],colorsxlab="",ylab="")
#yellowish colours in this plot are where intensities in this plot are very high
graphics::image
heat.colors(12)
#use nifti instead of R , r treats different types of objects differently
image(nii_T1,z=11,plot.type="single")
oro.nifti::image
#plot all slices - but it wont like that
#image(nii_T1)
#all planes coronal,axial
orthographic(nii_T1,xyz=c(200,220,11))
par(mfrow=c(1,2));
o<-par(mar=c(4,4,0,0))
hist(nii_T1,breaks=75,prob=T,xlab="T1 intensities", col=rgb(0,0,1,1/2),main="");
#have to do >20 get rid of all black area which is meaningless
hist(nii_T1[nii_T1 > 20],breaks = 75, prob=T,xlab="T1 intenties > 20", col=rgb(0,0,1,1/2),main="")
\end{verbatim}

\begin{figure}
\centering
\includegraphics{C:/Neurohacking/histograms.jpeg}
\caption{histograms}
\end{figure}

\begin{verbatim}
#create a mask for particular value intensites
#this is the white matter of the brain where the axons are in the brains and we see the skull
is_btw_300_400 <- ((nii_T1>300) & (nii_T1<400))
nii_T1_mask<-nii_T1
nii_T1_mask[!is_btw_300_400]=NA
overlay(nii_T1,nii_T1_mask,z=11,plot.type="single")
#can do all but this breaks my commputer
#overlay(nii_T1,nii_T1_mask)
#can also do orthographic
orthographic(nii_T1,nii_T1_mask,xyz=c(200,2,2,11), text="image overlaid with mask",text.cex=1.5)
\end{verbatim}

\begin{figure}
\centering
\includegraphics{C:/Neurohacking/whitematter.jpeg}
\caption{whitematter}
\end{figure}

\begin{verbatim}

#read data 

mridir <- "C:/Neurohacking/Neurohacking_data/kirby21/visit_1/113"
T1 <- readNIfTI(file.path(mridir,"/113-01-MPRAGE.nii"),reorient=FALSE)
orthographic(T1)

#read in and view binary mask for the image

mask <- readNIfTI(file.path(mridir,"/113-01-MPRAGE_mask.nii"),reorient=FALSE)
orthographic(mask)
#must be in the same dimensions
#areas not in the mask are now gone
#got rid of bone area and things that arent actually brain
masked.T1 <- T1*mask
orthographic(masked.T1)

#take one image away from another 
#also can do multiply, modulus, exponential, log transform etc

T1.follow <- readNIfTI(file.path(mridir,'/113-02-MPRAGE.nii'),reorient=FALSE)
subtract.T1 <- T1.follow - T1
min(subtract.T1)
max(subtract.T1)
orthographic(subtract.T1)

im_hist<-hist(T1,plot=FALSE)
par(mar=c(5,4,4,4) + 0.3)
col1=rgb(0,0,1,1/2)
plot(im_hist$mids,im_hist$count,log="y",type="h",lwd=10,lend=2,col=col1,xlab="intentsity values",ylab="count (log)")
#linear transfomration
par(new=TRUE)
curve(x*1,axes=FALSE,xlab="",ylab="",col=2,lwd=3)
axis(side=4,at=pretty(range(im_hist$mids))/max(T1),labels=pretty(range(im_hist$mids)))
mtext("original intensity",side=4,line=2)
#transform them by differential, define a linear spline, break line in to knots
lin.sp<-function(x,knots,slope)
  {knots<-c(min(x),knots,max(x))
   slopeS<-slope[1]
  for(j in 2:length(slope)){slopeS<-c(slopeS,slope[j]-sum(slopeS))}
   
  rvals<-numeric(length(x))
  for(i in 2:length(knots))
    {rvals<-ifelse(x>=knots[i-1],slopeS[i-1]*(x-knots[i-1])+rvals,rvals)}
  return(rvals)}
knots.vals<-c(.3,.6)
slp.vals<-c(1,.5,.25) 
\end{verbatim}

\begin{figure}
\centering
\includegraphics{C:/Neurohacking/ortho.jpeg}
\caption{ortho}
\end{figure}

\begin{figure}
\centering
\includegraphics{C:/Neurohacking/ortho2.jpeg}
\caption{takeawaywhitematter}
\end{figure}


\end{document}
